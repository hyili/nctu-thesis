\begin{abstractzh}
BitFission 是一個整合了位元洪流協議與區塊層級再生的裸機服務開通框架,
透過對等式網路快速部署大量裸機伺服器。
應用於叢集運算中心或是電腦教室等高密度主機環境,
可以大幅降低作業系統與軟體部署時間。
相較於多播協議,使用位元洪流作為傳輸協議可以支援續傳以及異步傳輸,
容錯機制更加完善且便利。
而傳輸效率方面也不遜於多播傳輸,經實驗測試,在一般電腦教室甚至快於多播協議。
除了改善部署方式,BitFission 亦整合了 Ansible 進行無人值守組態設定,
降低人工操作成本,加速服務開通效率。\\

\noindent
關鍵字:裸機服務開通、位元洪流、組態管理系統、部屬系統、區塊層級再生
\end{abstractzh}



\begin{abstracten}
BitFission is a bare-metal provisioning framework intergrating BitTorrent protocal and block-level cloning.
With peer-to-peer transmission, we can fastly deploy numbers of bare-metal hosts.
In high-density bare-metal environments such as clustering computing centers or computer rooms,
BitFission significantly reduces time costs in deploying operating systems and applications.
Compared to Multicast protocal, using BitTorrent as a transmission protocal can do resuming and asyncronous transmission, which makes fault-tolerant mechanism more convenient and sound.
At transmission speed, BitTorrent is compatible with Multicast.
In our experiments, BitTorrent is even faster than Multicast in a computer room.
Despite of deployment, BitFission also intergated Ansible to archieve unattended configuration management,
which excludes manual operations and speeds up provisioning.\\

\noindent
Keywords: Bare-Metal Provisioning, BitTorrent, Configuration Management Tools, Deployment System, Block-level Cloning
\end{abstracten}
