\chapter{研究背景}
\section{Bare-Metal Provisioning}
縱使在虛擬化運算高度發展的現在,裸機運算仍有其不可取代的功能性與效能,也因此為了能夠降低裸機部署與管理的成本,裸機服務開通越顯重要。一個典型的服務開通架構,是由服務開通者(Provisioner)對主機群的各節點進行服務開通。
服務開通者可以透過人工或是自動化的方式將實體主機註冊成主機群的節點,節點被註冊後由服務開通者進行部署(Deployment)、組態管理(Configuration Management)與資源調節(Schedule)。
服務開通在裸機主機群上沒有辦法透過 Hypervisor 直接操作節點,因此會利用 IPMI 或是其他工具管理節點的電源與硬體資源。
此外,部署服務所需的映像檔也需要實體網路與硬體配合,一般而言節點使用 PXE 與服務開通者的 DHCP 伺服器溝通如何部署作業系統映像檔。
而服務的運作並非只需要作業系統,還有各項設定與管理,因此服務開通者會使用組態管理系統(Configuration Management System),在完成部署後在節點的作業系統上進行部署後設定(post-deployment configuration),讓節點能夠運行目標作業。


\section{Deployment System}
本研究發現近年的裸機服務開通框架\cite{chandrasekar2014comparative}沒有高效率的作業系統部署方式,大多仍是伺服器與主機間一對一傳輸映像檔的方式進行部署,相當倚賴部署伺服器的硬碟效能與網路頻寬。
如果依此方式傳輸映像檔部署時間會是 O(n) 線性成長,在大規模叢集下部署上千台機器所需的時間成本相當龐大,需要有加速部署的方法,因此本研究著手進行如何高效率部署映像檔。
在部署系統上的相關實作有:

\begin{enumerate}
\item 台灣國家高速網路與計算中心開發的 Clonezilla\cite{shiau2008clonezilla}。
\item 法國國立計算機及自動化研究院東部研究中心(INRIA Nancy)開發的 Kadeploy3\cite{kadeploy3}。
\end{enumerate}


Clonezilla 透過 udpcast 實作了 Multicast-based broadcast 部署方法,理論上在同一個 Local Area Network (LAN) 底下部署可以達到 O(1) 常數時間的複雜度。
然而傳輸每一個封包的失敗率會因為機器數量而上升,導致封包需要重新傳送才能同步狀態。根據我們在交大資工系計中進行的部署實驗,在 50 台主機的規模下,傳輸速度只有 30 MB/s,且機器部署失敗率達 6\%。此外,Clonezilla 的部署規模僅適用於同一個 LAN 底下,不適用於大規模的叢集。


Kadeploy3 透過 TakTuk 實作兩個部署方法 Tree-based broadcast 和 Chain-based broadcast。TakTuk 原理是透過路徑搜尋演算法,計算出樹狀結構或是鏈狀結構連結所有節點的網路拓樸。根據他們自身的實驗數據\cite{sarzyniec2012scalability},僅部署 430 MB 的映像檔在 3999 台主機,有 161 台主機無法完成部署,約為 4\% 的失敗率,費時 57 分鐘。本研究認為此方法在記憶體快取上的利用不足,且失敗率仍有改善的空間。除此之外 Kadeploy3 是透過記憶體暫存映像檔,載入完整映像檔後才解壓縮至硬碟中,受限於記憶體大小。


由於上述理由,本研究認為裸機服務開通需要有更快速且穩定的映像檔部署方法,且能夠配合各種規模的網路架構。

\section{Image Transmission}
部署效率上,影響最大的因子正是映像檔傳輸的速度,一個輕量的伺服器映像檔可能就有 10GB 的容量,以此進行上百台機器的部署,如何加速是一個關鍵的議題。
普遍的主從式傳輸仰賴著伺服器本身的頻寬,無法應用於大規模部署,因此需要使用其他的傳輸方式。
網路傳輸協定發展至今,IP 協定主要有兩個多目標傳輸資料的技術:

\begin{enumerate}
\item Multicast

Multicast 主要使用 UDP 協定進行傳輸,在每個網路節點上進行複製,傳輸到節點的下游目標主機。
然而 UDP 協定本身不具備可靠性,因此有發展可靠版本的 Multicast 技術,如 RTPS 或是 PGM。
儘管理論上 Multicast 使用在多目標傳輸模型上的效能是完美的,實際上隨著傳輸規模增加,Multicast 的效意卻大幅下降,需要極大的網路成本。
在裸機部署上,更要考量到不同機器間硬體的延遲與佇列排滿丟失的問題,Multicast 必須維持所有目標的一致性,效能隨著目標數下滑,因此 Multicast 難以直接應用於大規模部署。
\item Peer-to-Peer

Peer-to-Peer (P2P) 利用每個目標主機的資源,彼此傳輸資料,大幅降低部署伺服器頻寬的需求,且每個主機皆提供自身的資源作為快取或是傳輸,因此目標數增加對於傳輸效率的影響,並不如 Multicast 般嚴重。
然而現有使用 P2P 技術的部署系統無法直接對硬碟寫入映像檔,需要使用記憶體儲存,受限於記憶體空間,也不支援區塊層級的再生,因此本研究希望開發出支援前兩者的 P2P 部署系統。
\end{enumerate}


\section{Operating System Image}
為了要執行運算的程式,每個主機都需要有作業系統,而如何規劃作業系統映像檔是裸機服務開通中相當重要的議題。
如 Texas Advanced Computing Center 的研究\cite{mclay2011best}中,他們使用了 Cobbler 部署薄映像檔(thin image),僅包含 Linux 作業系統與基本設定,之後再透過作業系統案裝運算用的應用程式。
但是 INRIA 在 kadeploy3 的研究中\cite{kadeploy3}指出薄映像檔並不適合多用戶的大型叢集運算中心,因此決定讓用戶提供包含應用程式與基礎設定的厚映像檔(thick image),再透過 Kadeploy3 進行部署。
此外,厚映像檔相較於薄映像檔還有以下優勢:
\begin{enumerate}
\item 厚映像檔花費較多時間在製作過程,但減少了部署時間。
\item 厚映像檔可以在部署前驗證正確性。
\item 厚映像檔確保部署目標主機的一致性。
\item 部分軟體缺乏無人值守的安裝流程,需要人工執行安裝程式。
\end{enumerate}
因此在交大資工系計中,我們為了要提供課程用的作業系統,必須包含許多不同設定的應用程式,安裝的過程複雜且無法自動化,不同程式之間又有可能影響彼此運作。因此我們決定使用部署厚映像檔的方式,事先驗證正確性,避免安裝過程的疏漏影響系統運作。
