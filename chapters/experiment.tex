\chapter{實驗結果}
\label{c:experiment}
\section{實驗環境}
本研究實驗大多於交大資工計算機中心電腦教室完成,其電腦配備如表 \ref{pcspec}。
\begin{table}[!htbp]
\centering
\caption{實驗主機群硬體規格}
\label{pcspec}
\begin{adjustbox}{max width=0.92\textwidth}
\begin{tabular}{lrrrrrr}

\toprule
\multicolumn{1}{l}{\textbf{主機群}} & \textbf{EC315} & \textbf{EC316} & \textbf{EC324} \\ \midrule
\multicolumn{1}{l}{\textbf{主機數目}} & \textbf{70} & \textbf{60} & \textbf{50} \\

\multicolumn{1}{l}{\textbf{CPU}} & \textbf{i5-4570s 2.9GHz} & \textbf{i5-2400s 2.49GHz} & \textbf{i5-7500 3.4GHz} \\

\multicolumn{1}{l}{\textbf{記憶體}} & \textbf{4GB DDR3} & \textbf{4GB DDR3} & \textbf{16GB DDR4} \\

\multicolumn{1}{l}{\textbf{傳統硬碟}} & \textbf{1TB} & \textbf{500GB} & \textbf{1TB} \\

\multicolumn{1}{l}{\textbf{固態硬碟}} & \textbf{N/A} & \textbf{N/A} & \textbf{250GB} \\

\multicolumn{1}{l}{\textbf{網路頻寬}} & \textbf{1Gb/s} & \textbf{1Gb/s} & \textbf{1Gb/s} \\
\bottomrule
\end{tabular}
\end{adjustbox}
\end{table}

\section{實驗目的}
本研究一共進行了四項實驗:
\begin{enumerate}
\item{裸機再生實驗}

透過與 Clonezilla S.E. 版本的 Partclone 進行再生速度之比較,期望在同時計算種子檔雜湊值的情況下,BitFission 版本的 Partclone 仍能順利完成映像檔製作。

\item{映像檔部署實驗}

本研究針對 HDD 與 SSD 分別進行 BitTorrent 與 Multicast 部署效率之比較,期望 BitTorrent 在電腦教室之規模下,能夠勝過 Multicast 之傳輸速率。

\item{容錯機制測試}

BitTorrent 因協議本身較 Multicast 容錯性高,本研究以網路分區(partition)進行容錯測試,期望 BitFission 能夠達成 Multicast 無法處理的網路中斷或是無回應主機等情況。

\item{系統組態設定實驗}

本實驗針對交大系計中電腦教室組態設定自動化,期望能夠在無人值守之情況,全程透過 ansible 遠端將部署完成之主機自動初始化,提供課程與學生使用之環境。

\end{enumerate}
\section{裸機再生實驗}
\begin{table}[htbp]
\centering
\caption{SSD 映像檔再生實驗}
\label{pcspec}
\begin{adjustbox}{max width=0.92\textwidth}
\begin{tabular}{lcccccc}

\toprule
\multicolumn{1}{l}{\textbf{再生軟體}} & \textbf{BitFission 版本 Partclone} & \textbf{Clonezilla 版本 Partclone} \\ \midrule
\multicolumn{1}{l}{\textbf{分割區使用量}} & \multicolumn{2}{c}{\textbf{194.6GB}} \\

\multicolumn{1}{l}{\textbf{硬碟類型}} & \multicolumn{2}{c}{\textbf{250GB SSD}} \\

\multicolumn{1}{l}{\textbf{CPU}} & \multicolumn{2}{c}{\textbf{CPU i5-7500 3.4GHz}} \\

\multicolumn{1}{l}{\textbf{儲存空間}} & \multicolumn{2}{c}{\textbf{遠端網路檔案系統伺服器(NFS)}} \\

\multicolumn{1}{l}{\textbf{網路頻寬}} & \multicolumn{2}{c}{\textbf{1Gbps}} \\

\multicolumn{1}{l}{\textbf{種子檔雜湊計算}} & \textbf{啟用} & \textbf{無} \\

\multicolumn{1}{l}{\textbf{壓縮演算法}} & \textbf{lz4(多核心)} & \textbf{gzip(多核心)} \\

\multicolumn{1}{l}{\textbf{耗費時間}} & \textbf{27分鐘27秒} & \textbf{39分鐘53秒} \\

\multicolumn{1}{l}{\textbf{再生速度}} & \textbf{7.08GB/min} & \textbf{4.88GB/min} \\

\multicolumn{1}{l}{\textbf{壓縮映像檔大小}} & \textbf{113 GB} & \textbf{105 GB} \\

\bottomrule
\end{tabular}
\end{adjustbox}
\end{table}

\begin{table}[!htbp]
\centering
\caption{HDD 硬碟再生實驗}
\label{imaging_hdd}
\begin{adjustbox}{max width=0.92\textwidth}
\begin{tabular}{lcccccc}

\toprule
\multicolumn{1}{l}{\textbf{再生軟體}} & \textbf{BitFission 版本 Partclone} & \textbf{Clonezilla 版本 Partclone} \\ \midrule
\multicolumn{1}{l}{\textbf{分割區使用量}} & \multicolumn{2}{c}{\textbf{51.7GB}} \\

\multicolumn{1}{l}{\textbf{硬碟類型}} & \multicolumn{2}{c}{\textbf{1TB SSD}} \\

\multicolumn{1}{l}{\textbf{CPU}} & \multicolumn{2}{c}{\textbf{CPU i5-7500 3.4GHz}} \\

\multicolumn{1}{l}{\textbf{儲存空間}} & \multicolumn{2}{c}{\textbf{遠端網路檔案系統伺服器(NFS)}} \\

\multicolumn{1}{l}{\textbf{網路頻寬}} & \multicolumn{2}{c}{\textbf{1Gbps}} \\

\multicolumn{1}{l}{\textbf{種子檔雜湊計算}} & \textbf{啟用} & \textbf{無} \\

\multicolumn{1}{l}{\textbf{壓縮演算法}} & \textbf{lz4(多核心)} & \textbf{gzip(多核心)} \\

\multicolumn{1}{l}{\textbf{耗費時間}} & \textbf{9分鐘30秒} & \textbf{15分鐘42秒} \\

\multicolumn{1}{l}{\textbf{再生速度}} & \textbf{5.56GB/min} & \textbf{3.29GB/min} \\

\multicolumn{1}{l}{\textbf{壓縮映像檔大小}} & \textbf{38 GB} & \textbf{37 GB} \\

\bottomrule
\end{tabular}
\end{adjustbox}
\end{table}

\section{映像檔部署實驗}
\begin{table}[htbp]
\centering
\caption{映像檔部屬實驗一}
\label{pcspec}
\begin{adjustbox}{max width=0.92\textwidth}
\begin{tabular}{lcccccc}

\toprule
\multicolumn{1}{l}{\textbf{部屬系統}} & \textbf{BitFission BT server} & \textbf{Clonezilla SE - multicast} \\ \midrule
\multicolumn{1}{l}{\textbf{實驗主機群}} & \multicolumn{2}{c}{\textbf{EC324}} \\
\multicolumn{1}{l}{\textbf{實驗主機數}} & \multicolumn{2}{c}{\textbf{45}} \\
\multicolumn{1}{l}{\textbf{CPU}} & \multicolumn{2}{c}{\textbf{i5-7500 3.4GHz}} \\
\multicolumn{1}{l}{\textbf{記憶體}} & \multicolumn{2}{c}{\textbf{16GB DDR4}} \\
\multicolumn{1}{l}{\textbf{網路頻寬}} & \multicolumn{2}{c}{\textbf{1Gbps}} \\
\multicolumn{1}{l}{\textbf{硬碟類別}} & \multicolumn{2}{c}{\textbf{1TB HDD}} \\
\multicolumn{1}{l}{\textbf{還原容量}} & \multicolumn{2}{c}{\textbf{194.6GB}} \\
\multicolumn{1}{l}{\textbf{耗費時間}} & \textbf{95分鐘} & \textbf{125分鐘} \\
\multicolumn{1}{l}{\textbf{還原速度}} & \textbf{2.05GB/min} & \textbf{1.55GB/min} \\

\bottomrule
\end{tabular}
\end{adjustbox}
\end{table}

\input{tables/experiment_deployment2.tex}
\begin{table}[!htbp]
\centering
\caption{映像檔部署實驗之三}
\label{exp3}
\begin{adjustbox}{max width=0.92\textwidth}
\begin{tabular}{lcccccc}

\toprule
\multicolumn{1}{l}{\textbf{部屬系統}} & \textbf{BitFission BT server} & \textbf{Clonezilla SE - multicast} \\ \midrule
\multicolumn{1}{l}{\textbf{實驗主機群}} & \multicolumn{2}{c}{\textbf{EC324}} \\
\multicolumn{1}{l}{\textbf{實驗主機數}} & \multicolumn{2}{c}{\textbf{49}} \\
\multicolumn{1}{l}{\textbf{CPU}} & \multicolumn{2}{c}{\textbf{i5-7500 3.4GHz}} \\
\multicolumn{1}{l}{\textbf{記憶體}} & \multicolumn{2}{c}{\textbf{16GB DDR4}} \\
\multicolumn{1}{l}{\textbf{網路頻寬}} & \multicolumn{2}{c}{\textbf{1Gbps}} \\
\multicolumn{1}{l}{\textbf{硬碟類別}} & \multicolumn{2}{c}{\textbf{1TB HDD}} \\
\multicolumn{1}{l}{\textbf{還原容量}} & \multicolumn{2}{c}{\textbf{73.4GB}} \\
\multicolumn{1}{l}{\textbf{映像檔位置}} & \multicolumn{2}{c}{\textbf{部署伺服器硬碟(SSD)}} \\
\multicolumn{1}{l}{\textbf{耗費時間}} & \textbf{34分鐘} & \textbf{57分鐘} \\
\multicolumn{1}{l}{\textbf{還原速度}} & \textbf{2.15GB/min} & \textbf{1.28GB/min} \\

\bottomrule
\end{tabular}
\end{adjustbox}
\end{table}

\section{容錯機制測試}
\label{faulttolerant}
本研究使用三台虛擬機器,分別對 BitFission 與 Clonezilla S.E. 進行容錯測試,測試項目有:
\begin{enumerate}
\item 切斷網路連線後恢復

BitFission 切斷其中一台網路後,未暫停的虛擬機仍繼續部署作業,而網路恢復後需重新跟 Tracker 連線找尋同儕接續部署。而 Clonezilla S.E. 切斷其中一台網路後,未切斷網路者停滯並等待,等至網路恢復後才接續部署。
\item 暫停虛擬機器後恢復

BitFission 暫停其中一台虛擬機後,未暫停的虛擬機仍繼續部署作業,而暫停機器恢復後順利接續部署。而 Clonezilla S.E. 暫停其中一台虛擬機後,未暫停的虛擬機停滯並等待,再將暫停機器恢復後才開始繼續部署。
\item 重新啟動虛擬機器

BitFission 的客戶端在重新啟動後,再次執行的還原腳本,順利地與 Tracker 連線並找到其他同儕重新進行部署。而 Clonezilla S.E. 重啟其中一台虛擬機後,因為進度不一致,使得 Multicast 無法接續進行,必須重新進行部署。
\end{enumerate}
\section{系統組態設定實驗}
\begin{lstlisting}[label={ansible},caption= Shell commands of ansible-playbook]
ansible-playbook --extra-vars="var_host=ec324" /etc/ansible/playbook/post_deploy_stage1.yml 
ansible-playbook --extra-vars="var_host=ec324" /etc/ansible/playbook/post_deploy_stage2.yml
\end{lstlisting}
本研究於映像檔部署後進行系統組態設定,透過 ansible-playbook 平行化進行設定,實際指令如原始碼 \ref{ansible}。
post\_deploy\_stage1.yml 包含了對應 IP 位置修改主機名稱、Windows 作業系統 KMS 認證、更新印表機資訊和加入 Windows AD 網域。執行完畢後會重新啟動等待下一階段設定。
post\_deploy\_stage2.yml 包含了更變 SID 數值、更新 GPO 政策和一般軟體授權。組態設定完成後就能將機器上線運行,全程無人值守並於半小時內完成。
