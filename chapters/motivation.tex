\chapter{研究動機}
\section{應用環境}
\section{映像檔再生}
在交大資工系計算中心電腦教室中,我們提供了一共 180 台實體主機給系上師生和課程使用。
然而每門課程的使用需求不盡相同,需要不同的應用程式或作業系統,因此一台主機中會包含上萬個檔案且佔用數百 GB 的硬碟空間。若採用檔案層級再生(file-level cloning)的映像檔不僅會影響部屬效率,也無法確保部屬後實體主機的一致性,可能會有異常或是檔案毀損的問題存在。
因此本研究認為區塊層級再生(block-level cloning)的映像檔在裸機服務開通系統上是重要的功能,在開發 BitFission 上採用區塊層級再生來製作映像檔。
\section{映像檔部屬}
以往在電腦教室中,我們透過最普遍的多播傳輸部屬映像檔到各台主機上。
但是多播傳輸必須確保每一台主機在部屬進度上的一致性,如果發生單台主機失去回應(unresponsive)便會阻塞多播傳輸。
尤其在頻繁使用的電腦教室中,硬體老化提高了單台主機故障的機率,大幅降低了多播傳輸的效率。
此外,即使沒有主機故障或是失去回應,仍有許多因素會造成多播傳輸的速度下滑,例如單台主機的硬碟寫入延遲。
一般而言,我們需要耗費相當的人力去偵錯,因為多播傳輸的一致性,讓熱點(hotspots)難以被判別。
因此在裸機服務開通上,本研究認為需要有更加快速且更可靠的傳輸方法,解決上述各項部屬的問題。
\section{組態設定}
基於課程需求,我們大多數的映像檔是使用 Windows 作業系統,但並不受限於之。
即便可以透過 Windows AD 或是 winrm 等技術遠端進行組態設定,在多台主機上進行狀態控制與例外處理仍需要人工作業。
本研究期望有一個無人值守的組態設定流程以及即時的狀況監控,如今許多跨平台的開源組態管理工具可以達到上述的需求。
因此本研究分析了一些組態設定工具,決定何者適合做為 BitFission 的組態管理系統核心工具。

\section{研究目標}
綜合以上所有動機,本研究有以下目標:
\begin{enumerate}
\item 區塊層級再生的映像檔
\item 容錯無回應主機的部屬機制
\item 可規模化的映像檔傳輸方法
\item 無人值守的部屬流程
\item 基於開放原始碼專案的裸機服務開通框架
\end{enumerate}

