\chapter{研究緒論}
\label{c:intro}
近來人工智慧與大數據科學的興起,在資訊科學領域的研究中,為了進行訓練以及資料分析,對於高速運算(High Performance Computing)的需求日益顯耀。然而虛擬化伺服器沒有辦法提供如同實體主機(bare-metal hosts)一般的效能,而實體主機群也沒有 Hypervisor 能夠輕易管理大規模的叢集(cluster),因此研究如何部署與管理裸機叢集(bare-metal clusters)的裸機服務開通(bare-metal provisioning)變成相當重要的議題。
然而目前用於裸機服務開通的 OpenStack Ironic 或是 Microsoft Deployment Toolkit 等部屬軟體,沒有辦法提供規模化部署的解決方案,僅能透過伺服器與主機間主從式傳輸映像檔的方式進行部署。
為了解決大規模實體機器部屬的問題,必須要採用不同的傳輸方法才能夠快速的部署多台實體主機。
以往在這方面的相關實作是使用 multicast 技術的專案,例如國家高速網路與計算中心開發的硬碟還原工具 Clonezilla\cite{shiau2008clonezilla} 使用 udpcast 實作了 multicast 部署系統,可部屬多達兩百台的實體主機群。
然而 multicast 非但有網路架構的限制,即使在良好的網路環境下,因其 UDP 的限制造成封包遺失以及同步困難的問題,無法穩定完成所有機器的部署。
因此為了達到快速、穩定、自動化的三大需求,本研究將以交通大學資訊工程學系計算機中心部署方案 BitFission 為例,研究如何利用基於 BitTorrent 協議的 EZIO 以及開放原始碼的組態管理工具 (configuration management tools) Ansible 完成自動化部署與設定電腦教室主機群。

BitFission 是一個基於 BitTorrent 協議和開源專案的裸機服務開通框架。
在映像檔製作上,本研究整合了 Partclone、EZIO 和 Squashfs 等開源專案,讓 BitFission 得以將實體機器轉換成區塊等級(block-level)的壓縮映像檔,並且同時製作種子檔用於之後部屬使用。
在機器部屬上,本研究結合了 Wake-On LAN 、PXE 開機和 pxelinux ,完成一個無人值守的部屬流程。
在映像檔傳輸上,本研究採用了 BitTorrent 協議和 EZIO 專案,使得映像檔傳輸相較其他傳輸方法更加快速且更加可靠。
在組態設定上,本研究使用了 Ansible 建立一套組態管理系統。
在 45 台主機部屬實驗中,multicast 模式的 Clonezilla SE 的傳輸速度為每台 1.55 GB/min,而 BitFission 的傳輸速度則是 2.05 GB/min,約 32\% 的效能提升。更進一步,BitFission 相較於 Clonezilla SE 可以容錯暫時的分區 (temporary partitions),並不會受到嚴重的效能損益。從映像檔製作到完成組態設定,BitFission 在 154 分鐘內完成 194.7 GB 映像檔的部屬 45 主機的任務。


