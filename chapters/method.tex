\chapter{研究方法}
\section{研究設計與流程}
由於 BitFission 目標主機群是 PC 電腦教室,因此將透過 Wake-On-Lan、PXE 和 Ansible,進行 Bare-metal 設定與操作,即使沒有 IPMI 可以遠端操作的功能,BitFission 仍可以完成目標主機群的開關機與自動化部屬。


為了解決 Server-Client 一對一效能問題以及 Multicast 封包遺失所造成的不穩定,BitFission 使用了基於 BitTorrent 協議的 EZIO 作為映像檔部屬的解決方案。因為 BitTorrent 透過 P2P 讓部屬速度接近 Multicast ,在特定情況下甚至可以超過 Multicast。且 BitTorrent 具備非同步的特性,部屬時不受單一主機異常而影響其他主機,異常主機也可以在一定時間內重新回到部屬作業中。


此外,由於 BitTorrent 的 resume 功能,使得部屬作業得以僅還原硬碟與映像檔有差異的片段 ( pieces ),進行差異式部屬降低硬碟寫入次數與網路傳輸量。對於使用同一基礎映像檔 ( Base Image ) 產生的差異映像檔 ( Differential Image ),可以提高部屬效率並減少硬碟寫入次數。


Ansible 是一個開源的組態管理工具,作為 BitFission 後續設定的解決方案。選用 Ansible 原因是由於 agentless 架構以及其連線機制是透過 ssh 或 winrt ,可以自由的整合不同作業系統,包含 Unix-like 和 Windows 作業系統。此外,Ansible 也不用在部屬時在映像檔加入憑證,如 Puppet 是透過憑證認證,每台主機必須事先產生不同憑證才能夠與 Puppet 進行後續設定。綜合以上優點,我們使用 Ansible 作為 Bare-Metal 主機群的組態管理工具。


為了自動化產生映像檔,透過 PXE 開機載入 bootable linux image,執行 Partclone 將原型機的檔案系統轉換成映像檔。部屬伺服器使用 BitTorrent 協議分析映像檔製作 torrent ,並以部署伺服器作為 tracker 和第一個 seeder 透過 P2P 進行部屬。之後使用 Wake-On-LAN 喚醒目標主機群,以 PXE 開機載入 S. EZIO 系統。 S. EZIO 系統透過 BitTorrent 協議和 tracker 溝通後找到 seeder ,執行 EZIO 程式與 seeder 交換映像檔並寫入硬碟中完成映像檔部屬。映像檔部屬後所需的後續設定 ( Post-Configuration ) ,BitFission 採用 Ansible 調整每台主機的差異部分,例如修改主機名稱與 Join Domain 等等動作。以此架構達成自動化 Bare-Metal 機器部屬與服務開通 ( Provisioning )。

\begin{comment}
本研究吉祥物如圖~\ref{i:cat}。
\input{figures/cat}

樹範例如~\ref{i:tree}
\input{figures/tree}

長條圖範例如~\ref{i:barchart}
\input{figures/barchart}


本研究分組如表~\ref{t:group}
\input{tables/group}
\end{comment}





    






