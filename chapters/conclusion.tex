\chapter{結論與建議}
\label{c:conclusion}
\section{總結}
在實驗中,因為 BitFission 使用 lz4 壓縮法,
製作映像檔較原生 Partclone 更有效率,提升了 45 個百分點(表 \ref{imaging_ssd})。
在部署階段,BitFission 傳輸效率亦勝過 Clonezilla S.E. 的 Multicast 部署方式,
加速了 67 個百分點(表 \ref{exp3})。
而 BitFission 所使用的 BitTorrent 協議在容錯機制上,原生就比 Multicast 完善。
在容錯實驗中 [\ref{faulttolerant}],
亦顯示 BitTorrent 更妥善處理網路分區(partition)發生的情況。
總結上述實驗結果,
BitFission 展現了其優秀的部署效率以及高容錯的傳輸機制。
相較於其他裸機部署系統無法續傳或是區塊層級再生,
本研究在 BitFission 中達成了以上功能,
使得裸機服務開通更加穩定與快速。
因此本研究認為透過等對式網路進行裸機部署,
並透過組態管理工具進行部署後設定,
可以有效運用各節點之硬體資源與頻寬,
並達到無人值守部署與設定的目標,
大幅地減少了裸機服務開通所需要的時間與人力成本。

\section{未來建議}
目前 BitFission 對於不同硬體的支援相較 Clonezilla S.E. 少,
且 Clonezilla S.E. 相當成熟,擁有多種語言翻譯以及圖形化介面。
本研究可以與國家高速網路中心合作,
將 BitFission 的創新技術整合進 Clonezilla S.E.,
如此一來,Clonezilla S.E. 原先廣大的使用者亦可以體會對等式網路部署的效率。
